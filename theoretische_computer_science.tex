\documentclass[a4paper,11pt,utf8]{article}

\usepackage[utf8]{inputenc}
\usepackage[T1]{fontenc}
\usepackage{amsmath, amssymb, amsthm}
\usepackage{mathtools}
\usepackage{graphicx}
\usepackage{hyperref}
\usepackage{CJKutf8}
\usepackage{ctex}
\usepackage{geometry}
\usepackage{fancyhdr}
\usepackage{titling}
\usepackage{enumitem}
\usepackage{mdframed}
\usepackage{tcolorbox}
\usepackage{tikz}
\usepackage{forest}
\usepackage{multicol}

\usetikzlibrary{arrows.meta, positioning, shapes.geometric, calc, quotes, trees}

\renewcommand{\contentsname}{\text{Contents 目录}}
\renewcommand{\headrulewidth}{0.4pt}
\renewcommand{\footrulewidth}{0.4pt}

\newcommand{\newindent}{\\ \hspace*{\parindent}}
\newcommand{\lineindent}{\hspace*{\parindent}}


\geometry{
    left=1cm,
    right=1cm,
    top=1.5cm,
    bottom=1.5cm,
    includehead,
    includefoot
}

\title{\text{Theoretical Computer Science Summary}}
\author{玄风}
\newcommand{\copyrighttext}{\text{©玄风 2025 ALL RIGHTS RESERVED.}}
\date{}

\setlength{\headheight}{13.6pt}
\pagestyle{fancy}
\fancyhf{}
\fancyhead[R]{\thepage}
\fancyhead[L]{\nouppercase{\leftmark}}
\fancyfoot[L]{\text{Theoretical Computer Science}}
\fancyfoot[R]{\text{Updated: \today}}


\pretitle{\begin{center}\Huge\bfseries}
    \posttitle{\par\end{center}}
    \preauthor{\vfill\begin{center}\large}
    \postauthor{
      \\[1em]
      \normalfont\scriptsize \copyrighttext
      \end{center}
    }

\begin{document}

\maketitle
\thispagestyle{empty}
\newpage
\pagenumbering{roman}
\tableofcontents
\thispagestyle{empty}
\newpage

\clearpage
\pagenumbering{arabic}
\setcounter{page}{1} 

% 正文从此开始

\section{Basic Definition 基础概念}
\subsection{Alphabet, word and Language 字母表、单词和语言}
\begin{itemize}
    \item Alphabet $\Sigma$: $\Sigma_1 = \{a,b,c\}$, $\Sigma_2 = \{0,1\}$.
    \item word over $\Sigma$: $w_1 = aabc$ and $w_2 = cbac$ over $\Sigma_1$, $w_3 = 0001$ and $w_4 = 1001$ over $\Sigma_2$.
    \item word length $|w|$: $|w_1| = 4$, $|w_2| = 4$, $|w_3| = 4$, $|w_4| = 4$.
    \item Empty word $\lambda$: $\lambda$ is not a alphabet symbol, $|\lambda| = 0$.
    \item Set of all words over $\Sigma$ is $\Sigma^*$: $\{0,1\}^* = \{\lambda, 0, 1, 00, 01, 10, 11, 000, \dots\}$
    \item Formal language over $\Sigma$ is a subset of $\Sigma^*$: $L = \{00,10\} \subseteq \Sigma^* = \{0,1\}^*$.
    \item Empty language $\emptyset$ is a language without word: $\emptyset \neq \{\lambda\}$.
    \item Cardinality of a language $||L||$ is the number of word from $L$: $||L|| = 2$ for $L = \{00,10\}$, $||L|| = 0$ for $\emptyset$.
\end{itemize}
\subsection{Operations of words and languages 单词和语言的运算}
A and B are two languages over $\Sigma$. Aka $A,B \subseteq \Sigma^*$.
\begin{itemize}
    \item Union: $A \cup B = \{x \in \Sigma^* | x \in A \text{ or } x \in B\}$, e.g.: $\{10,1\} \cup \{1,0\} = \{10,1,0\}$.
    \item Intersection: $A \cap B = \{x \in \Sigma^* | x \in A \text{ and } x \in B\}$, e.g.: $\{10,1\} \cap \{1,0\} = \{1\}$.
    \item Difference: $A - B = \{x \in \Sigma^* | x \in A \text{ and } x \notin B\}$, e.g.: $\{10,1\} - \{1,0\} = \{10\}$.
    \item Complement: $\overline{A} = \Sigma^* - A = \{x \in \Sigma^* | x \notin A\}$, e.g.: $\overline{\{10,1\}} = \{0,1\}^* - \{10,1\} = \{\lambda,0,00,11,01,000, \dots\}$.
    \item Concatenation from words $u,v \in \Sigma^*$ is a word $uv \in \Sigma^*$, it defined:
    \subitem $u=v=\lambda$, $uv=vu=\lambda$.
    \subitem $v=\lambda$, $uv=u$.
    \subitem $u=\lambda$, $uv=v$.
    \subitem $u=u_1 u_2 \dots u_n$ and $v=v_1 v_2 \dots v_m$ with $u_i,v_i \in \Sigma$, so: $uv = u_1 u_2 \dots u_n v_1 v_2 \dots v_m$.
    \subitem e.g.: $u=01$, $v=10$, $uv=0110$, $vu=1001$.
    \subitem the language is: $AB = \{ab | a \in A \text{ and } b \in B\}$, e.g.: $\{10,1\}\{1,0\} = \{101,100,11,10\}$.
    \item Language iteration $A \subseteq \Sigma^*$ aka Kleene star is the language $A^*$, it defined: 
    \subitem $A^0 = \{\lambda\}$.
    \subitem $A^n = AA^{n-1}$.
    \subitem $A^* = \bigcup_{n \geq 0} A^n$
    \subitem $\lambda$-free iteration of A is $A^+ = \bigcup_{n \geq 1} A^n$.
    \subitem It exists: $A^+ \cup \{\lambda\} = A^*$, and it not exists: $A^+ = A^* - \{\lambda\}$.
    \subitem e.g.:
    \subitem $\{10,1\}^0 = \{\lambda\}$.
    \subitem $\{10,1\}^1 = \{10,1\}$.
    \subitem $\{10,1\}^2 = \{10,1\}\{10,1\} = \{1010,101,110,11\}$.
    \subitem $\{10,1\}^3 = \{10,1\}\{10,1\}^2 = \{101010, \dots\}$.
    \item Mirror image operation for a word $u = u_1 u_2 \cdots u_n \in \Sigma^*$ is defined with: $sp(u) = u_n \cdots u_2 u_1$, e.g.: $sp(0100101) = 1010010$.
    \item Language mirroring $A \subseteq \Sigma^*$ is defined with: $sp(A) = \{sp(w)|w \in A\}$, e.g.: $sp(\{100,011\}) = \{001,110\}$.
    \item Partial word relation of $\Sigma^*$ is defined with $u \sqsubseteq v \Longleftrightarrow (\exists v_1,v_2 \in \Sigma^*)[v_1 u v_2 = v]$.
    \subitem It exists $u \sqsubseteq v$, so $u$ is a partial word of $v$, aka infix. e.g.: $100 \sqsubseteq 1010010$.
    \item Initial word relation of $\Sigma^*$ is defined with $u \sqsubseteq_a v \Longleftrightarrow (\exists w \in \Sigma^*)[uw = v]$.
    \subitem It exists $u \sqsubseteq_a v$, so $u$ is a initial word of $v$, aka prefix. e.g.: $101 \sqsubseteq_a 1010010$.
\end{itemize}
\subsection{Grammar 语法}
Grammar is a quadraple $G = (\Sigma, N, S, P)$, where: 
\begin{itemize}
    \item $\Sigma$ is a alphabet, called terminal symbols.
    \item $N$ is a set of non-terminal symbols, $N \cap \Sigma = \emptyset$.
    \item $S \in N$ is a start symbol.
    \item $P \subseteq (N \cup \Sigma)^+ \times (N \cup \Sigma)^*$ is a terminal set of production rules.
\end{itemize}
\lineindent $(N \cup \Sigma)^+ = (N \cup \Sigma)^* - \{\lambda\}$.  \newindent
Rules $(p,q)$ in $P$ are defined with $p \to q$. \newindent
For grammar rules with same links side $A \in N$ has:
\[
\begin{array}{c}
A \to q_1 \mid q_2 \mid \cdots \mid q_n \\[1ex]
\text{\small BNF(Backus-Naur Form) notation}
\end{array}
\quad \text{ instead of } \quad
\begin{aligned}
A &\to q_1 \\
A &\to q_2 \\
& \vdots \\
A &\to q_n
\end{aligned}
\]
\subsection{Derivation relation, language of grammar 语法的推导关系和语言}
$G = (\Sigma, N, S, P)$ is a grammar, and $u$ and $v$ are words in $(N \cup \Sigma)^*$
\begin{itemize}
    \item Immediate derivation relation $\vdash$ is defined with: $u \vdash_G v \Longleftrightarrow u=xpz, v=xqz$, where $x,z \in (N \cup \Sigma)^*$ and $p \to q$ is a rule in $P$.
    \item By applying $n$ times of $\vdash_G$ has $\vdash_G^n$: $u = x_0 \vdash_G x_1 \vdash_G \cdots \vdash_G x_n = v$ for $n \geq 0$ and for a sequence from words $x_0, x_1, \dots, x_n \in (\Sigma \cup N)^*$, specially is $u \vdash_G^0 u$.
    \item The sequence $(x_0, x_1, \dots, x_n)$, $x_i \in (\Sigma \cup N)^*$, $x_0 = S$ and $x_n \in \Sigma^*$ is called a derivation tree, when $x_0 \vdash_G x_1 \vdash_G \cdots \vdash_G x_n$.
    \item Defined $\vdash_G^* = \Cup_{n \geq 0} \vdash_G^n$. It can be shouwn that $\vdash_G^*$ is the reflexive and transitive closure of $\vdash_G$, i.e. the smallest binary relation that is reflexive and transitive and encompasses $\vdash_G$.
    \item The language generated by the grammar $G$ is defined with: $L(G) = \{w \in \Sigma^* \mid S \vdash_G^* w\}$.
    \item Two grammars $G_1$ and $G_2$ are equivalent if $L(G_1) = L(G_2)$.
\end{itemize}
\subsection{Grammar, derivation, generated language 语法、推导和生成语言}
\begin{tcolorbox}[title=example 1,colback=white,colframe=black,width=\textwidth,arc=0pt]
    $G_1 = (\Sigma_1, \Gamma_1, S_1, R_1)$ is a grammar, where:
    \begin{itemize}
        \item terminal symbols $\Sigma_1 = \{a,b\}$.
        \item non-terminal symbols $\Gamma_1 = \{S_1\}$.
        \item the set of rules $R_1 = \{S_1 \to a S_1 b \mid \lambda\}$.
    \end{itemize}
    \lineindent the step of derivation for $G_1$ is:
    \[
        \begin{aligned}
            S_1 &\vdash_{G_1} a S_1 b \vdash_{G_1} a a S_1 b b \vdash_{G_1} aabb \\[1ex]
            S_1 &\vdash_{G_1}^1 a S_1 b \\[1ex]
            S_1 &\vdash_{G_1}^2 aa S_1 bb \\[1ex]
            S_1 &\vdash_{G_1}^3 aabb
        \end{aligned}
    \]
    \lineindent the derivation for $G_1$ is:
    \[
        S_1 \vdash_{G_1} a S_1 b \vdash_{G_1} aa S_1 bb \vdash_{G_1} aabb \Rightarrow aabb \in L(G_1).
    \]
    Apparently $G_1$ generates the language $L(G_1) = \{a^nb^n \mid n \geq 0\}$.
\end{tcolorbox}
\begin{tcolorbox}[title=example 2,colback=white,colframe=black,width=\textwidth,arc=0pt]
    $G_2' = (\Sigma_2, \Gamma_2', S_2', R_2')$ is a grammar, where:
    \begin{itemize}
        \item termimal alphabet is $\Sigma_2 = \{a,b,c\}$.
        \item non-terminal alphabet is $\Gamma_2' = \{S_2', B,C\}$.
        \item the set of rules is \[
            \begin{aligned}
                R_2' = \{ \quad & S_2' \to aB, \\[1ex]
                & B \to bC, \\[1ex]
                & C \to c S_2' \mid c \quad \}
            \end{aligned}
        \]
        \item $R_2' = \{S_2' \to aB, B \to bC, C \to cS_2', C \to c\}$.
        \item Apparently $G_2'$ generates the language $L(G_2') = \{(abc)^n \mid n \geq 1\}$.
    \end{itemize}
\end{tcolorbox}
\begin{tcolorbox}[title=example 3,colback=white,colframe=black,width=\textwidth,arc=0pt]
    $G_2 = (\Sigma_2, \Gamma_2, S_2, R_2)$ is a grammar, where:
    \begin{itemize}
        \item terminal alphabet is $\Sigma_2 = \{a,b,c\}$.
        \item non-terminal alphabet is $\Gamma_2 = \{S_2, B,C\}$.
        \item the set of rules is \[
            \begin{aligned}
                R_2 = \{ \quad & S_2 \to a S_2 BC \mid aBC, \\[1ex]
                & CB \to BC, \\[1ex]
                & aB \to ab, \\[1ex]
                & bB \to bb, \\[1ex]
                & bC \to bc, \\[1ex]
                & cC \to cc \quad \}.
            \end{aligned}
            \]
        \item $R_2 = \{S_2 \to a S_2 BC, S_2 \to aBC, CB \to BC, aB \to ab, bB \to bb, bC \to bc, cC \to cc\}$.
    \end{itemize}
    A derivation for $G_2$ is:
    \[
        \begin{aligned}
            S_2 &\vdash_{G_2} a S_2 BC  &&\vdash_{G_2} aa S_2 BCBC &&\vdash_{G_2} aaaBCBCBC \\[1ex]
                &\vdash_{G_2} aaaBBCCBC &&\vdash_{G_2} aaaBBCBCC   &&\vdash_{G_2} aaaBBBCCC \\[1ex]
                &\vdash_{G_2} aaabBBCCC &&\vdash_{G_2} aaabbBCCC   &&\vdash_{G_2} aaabbbCCC \\[1ex]
                &\vdash_{G_2} aaabbbcCC &&\vdash_{G_2} aaabbbccC   &&\vdash_{G_2} aaabbbccc
        \end{aligned}
    \]
    $\Rightarrow aaabbbccc \in L(G_2)$. \\
    \lineindent the generated language is: $L(G_2) = \{a^nb^nc^n \mid n \geq 1\}$.
\end{tcolorbox}
\begin{tcolorbox}[title=example 4,colback=white,colframe=black,width=\textwidth,arc=0pt]
    $G_3 = (\Sigma_3,\Gamma_3,S_3,R_3)$ is a grammar, where:
    \begin{itemize}
        \item terminal alphabet is $\Sigma_3 = \{*,+,(,),a\}$.
        \item non-terminal alphabet is $\Gamma_3 = \{S_3\}$.
        \item the set of rules is \[
            R_3 = \{S_3 \to S_3 + S_3 \mid S_3 * S_3 \mid (S_3) \mid a\}
        \]
    \end{itemize}
    A derivation for $G_3$ is:
    \[
        \begin{aligned}
            S_3 &\vdash_{G_3} S_3 + S_3 &&\vdash_{G_3} S_3 * S_3 + S_3 \\[1ex]
                &\vdash_{G_3} (S_3) * S_3 + S_3 &&\vdash_{G_3} (S_3 + S_3) * S_3 + S_3 \\[1ex]
                &\vdash_{G_3} \cdots &&\vdash_{G_3} (a + a) * a + a
        \end{aligned}
    \]
    \lineindent $\Rightarrow (a + a) * a + a \in L(G_3)$. \\
    \lineindent $G_3$ generates the language of all nested bracket expressions with the operations $+$ and $*$ and a character $a$.
\end{tcolorbox}
Notes:
\begin{itemize}
    \item the process of deriving words is inherently non-deterministic, because in general more than one rule can be applied in a derivation step.
    \item different grammars can produce the same language.
    \item in fact, every language that can be defined by a grammar has an infinite number of grammars that generate it.
    \item that is, a grammar is a syntactic object that generates a semantic object, namely its language.
    \item there are many different notations for grammars.
\end{itemize}
\subsection{Chomsky hierarchy 乔姆斯基层次}
\begin{tcolorbox}[title=Chomsky hierarchy,colback=white,colframe=black,width=\textwidth,arc=0pt]
    $G = (\Sigma,N,S,P)$ is a grammar, where:
    \begin{itemize}
        \item $G$ is type 0 grammar, aka unrestricted grammar, where $P$ is a set of rules with $|p| \geq 1$ and $|q| \geq 1$.
        \item $G$ is type 1 grammar, aka context-sensitive grammar, where $P$ is a set of rules $p \to q$ in $P$ with $|p| \leq |q|$.
        \item $G$ is type 2 grammar, aka context-free grammar, where $P$ is a set of rules $p \to q$ in $P$ with $p \in N$.
        \item $G$ is type 3 grammar, aka regular grammar, where $P$ is a set of rules $p \to q$ in $P$ with $p \in N$ and $q \in \Sigma \cup \Sigma N$.
        \item a language $A \subseteq \Sigma^*$ is from type $i \in \{0,1,2,3\}$, if there exists a type-$i$ grammar $G$ such that $L(G) = A$.
        \item the Chomsky hierarchy consists of the four language classes: \[
            \mathfrak{L}_i = \{L(G) \mid G \text{ is type } i \text{ grammar}\}, \text{with } i \in \{0,1,2,3\}.
        \]
        \item common names:
        \subitem $\mathfrak{L}_0$: recursively enumerable languages, aka Turing machines.
        \subitem $\mathfrak{L}_1$: context-sensitive languages, aka linear bounded automata. (CS)
        \subitem $\mathfrak{L}_2$: context-free languages, aka pushdown automata. (CF)
        \subitem $\mathfrak{L}_3$: regular languages, aka finite automata. (REG)
    \end{itemize}
\end{tcolorbox}
Fact: REG $\subseteq$ CF $\subseteq$ CS $\subseteq \mathfrak{L}$ \newindent
equivalent: type-3 $\subseteq$ type-2 $\subseteq$ type-1 $\subseteq$ type-0 $\subseteq$ all language
\begin{tcolorbox}[title=example,colback=white,colframe=black,width=\textwidth,arc=0pt]
    \begin{itemize}
        \item obviously, the $G_2$ and $G_3$ are context-sensitive grammars, because the length of the left-hand side of each production rule is less than or equal to the length of the right-hand side.
        \item the grammar $G_1$ has the rules $S_1 \to \lambda$ and is not context-sensitive because $\mid S_1 \mid = 1 > 0 = \mid \lambda \mid$ by definition (cf. "special rule for the empty word"). since $G_1$ is a grammar, $G_1$ is a type-0 grammar.
        \item obviously, the grammar $G_3$ from the previous example is even context-free.
    \end{itemize}
\end{tcolorbox}
\subsection{Special rules for the empty word 空字的特殊规则}
To include the empty word in languages of type-$i$ grammars for $i \in \{1,2,3\}$, make the following agreement:
\begin{itemize}
    \item the rule $S \to \lambda$ is the only shortening rule allowed for grammars of type $1,2,3$.
    \item if the rule $S \to \lambda$ occurs, $S$ must not occur on the right-hand side of any rule.
\end{itemize}
\lineindent This is not a restriction on the general public, because: \\ \lineindent
if there are rules in $p$ with $S$ on the right-hand side, we can change the rule set $P$ to the new rule set $P'$ as follows:
\begin{itemize}
    \item 1. in all rules of the form $S \to u$ from $P$ with $u \in (N \cup \Sigma)^*$, each occurrence of $S$ in $u$ is replaced by a new non-terminal $S'$.
    \item 2. in addition, $P'$ contains all rules from $P$, with $S$ replaced by $S'$.
    \item 3. the rule $S \to \lambda$ is added.
\end{itemize}
\lineindent The grammar $G' = (\Sigma, N \cup \{S'\}, S, P')$ modified in this way is (except for $S \to \lambda$) of the same type as $G$ and satisfies $L(G') = L(G) \cup \{\lambda\}$.
\begin{tcolorbox}[title=example,colback=white,colframe=black,width=\textwidth,arc=0pt]
    Given a grammar $G = (\Sigma, N, S, P)$ with:
    \begin{itemize}
        \item terminal alphabet $\Sigma = \{a,b\}$.
        \item non-terminal alphabet $N = \{S\}$.
        \item the set of rules $P = \{S \to aSb \mid ab\}$.
    \end{itemize}
    it's easy to see that $G$ is a context-free grammar and the language $L = \{a^nb^n \mid n \geq 1\}$ generated, i.e. $L(G) = L$. \newindent
    now modify the grammar $G$ according to the special rule for the empty word in order to obtain a context-free grammar for the language $\{a^nb^n \mid n \geq 0\} = L \cup \{\lambda\}$. \newindent
    so we got a context-free grammar $G' = (\Sigma, \{S,S'\}, S, P')$ with:
    \[
        \begin{aligned}
            P' = \{ \quad &S \to a S' b \mid ab, &&\text{according 1.}\\[1ex]
            &S' \to a S' b \mid ab, &&\text{according 2.}\\[1ex]
            &S \to \lambda \quad\}. &&\text{according 3.}
        \end{aligned}
    \]
    \lineindent and $L(G') = L(G) \cup \{\lambda\}$.
\end{tcolorbox}
\subsection{Syntex tree 语法树}
To visualize derivation of words in grammars of type 2 or 3, use a syntax tree defined as follows:
\begin{itemize}
    \item $G = (\Sigma, N, S, P)$ is a grammar from type 2 or 3, and $S = w_0 \vdash_G w_1 \vdash_G \cdots \vdash_G w_n = w$ is a derivation of $w \in L(G)$.
    \item the root (i.e. level 0) of the syntax tree is the start symbol $S$.
    \item level $i$: if, because of the rule $A \to z$ in the $i$-th derivation step ($w_{i-1} \vdash_G w_i$), the non-terminal $A$ in $w_{i-1}$ is replaced by the subword $z$ of $w_i$, the node $A$ in the syntax tree has $\mid z \mid$ sons, which are labeled from left to right with the symbols from $z$.
    \item the leaves of the tree, read form left to right, give $w$.
\end{itemize}
\begin{tcolorbox}[title=example,colback=white,colframe=black,width=\textwidth,arc=0pt]
    \begin{forest}
        [$S_3$
            [$S_3$
                [$S_3$
                    [$\text{(}$]
                    [$S_3$
                        [$S_3$ [a]]
                        [+]
                        [$S_3$ [a]]
                    ]
                    [$\text{)}$]
                ]
                [$*$]
                [$S_3$ [a]]
            ]
            [+]
            [$S_3$ [a]]
        ]
    \end{forest}
    the syntex for the derivation of words $(a+a)*a+a$ in the type-2 grammar $G_3$
\end{tcolorbox}
\subsection{Ambiguous grammar 模棱两可的语法}
Definition:
\begin{itemize}
    \item a grammar $G$ is called ambiguous if there is a word $w \in L(G)$ that has two different syntax trees.
    \item a language $A$ is called inherently ambiguous if every grammar $G$ with $A = L(G)$ is ambiguous.
\end{itemize}
\begin{tcolorbox}[title=example,colback=white,colframe=black,width=\textwidth,arc=0pt]
    \begin{forest}
        [$S_3$
            [$S_3$
                [$\text{(}$]
                [$S_3$
                    [$S_3$
                        [$S_3$ [a]]
                        [+]
                        [$S_3$ [a]]
                    ]
                ]
                [$\text{)}$]
            ]
            [$*$]
            [$S_3$
                [$S_3$ [a]]
                [+]
                [$S_3$ [a]]
            ]
        ]
    \end{forest}
    \newline
    \text{another syntax tree for the derivation of word $(a+a)*a+a$ in the type-2 grammar $G_3$.}
\end{tcolorbox}
\section{???}







% 正文结束

% \bibliographystyle{plain}
% \bibliography{references} % Add a references.bib file for bibliography

\end{document}
